
\documentclass[12pt, a4paper]{article}
\usepackage{hyperref}
\hypersetup{
  colorlinks=true,
  linkcolor=blue,
  urlcolor=cyan,
}
\urlstyle{same}
\usepackage[utf8]{inputenc}
\usepackage{amsmath}
\usepackage{amsfonts}
\usepackage{amssymb}
\usepackage{graphicx}


\newtheorem{theorem}{Teorema.}
\newtheorem{lemma}[theorem]{Lema.}
\newtheorem{corollary}[theorem]{Corolario.}
\newtheorem{definition}[theorem]{Definici\'on:}
\newtheorem{example}[theorem]{Ejemplo:}
\newtheorem{problema}[theorem]{Problema:}
\newtheorem{remark}[theorem]{Observaci\'on:}

\usepackage{graphicx}
\usepackage[spanish]{babel}
%\usetheme{default}

\newcommand{\pp}{\mathbb{P}}
\newcommand{\zz}{\mathbb{Z}}
\newcommand{\rr}{\mathbb{R}}
\newcommand{\qq}{\mathbb{Q}}

\usepackage{tikz, tikz-3dplot}

\definecolor{cof}{RGB}{219,144,71}
\definecolor{pur}{RGB}{186,146,162}
\definecolor{greeo}{RGB}{91,173,69}
\definecolor{greet}{RGB}{52,111,72}

\date{}

\begin{document}
\title{Pr\'actico ALGOBO: Búsqueda, Branch-and-bound y algoritmos genéticos.}
\author{Mauricio Velasco}
\maketitle{}
\begin{enumerate}
\item({\it Caníbales y Misionarios [Amarel, 1968]}) Tres misionarios y tres caníbales estan en la orilla sur de un río y quieren llegar a la orilla norte. Para cruzar tienen a su disposición un único barco capaz de llevar a lo más dos personas. Adicionalmente, el juego tiene las siguientes reglas:
\begin{enumerate}
\item Los caníbales se comerán a los misionarios si en cualquiera de las dos orillas del río hay más caníbales que misionarios.
\item El bote no puede cruzar el río sin un piloto.
\end{enumerate}
Encuentre una estrategia que permita transportar a los seis de un lado al otro del río de manera segura, resolviéndolo como un algoritmo de búsqueda.

\begin{enumerate}
\item Describa el espacio de estados.
\item Describa las decisiones válidas para cada estado.
\item Escriba su implementación de un algoritmo de búsqueda para resolver este problema (aclarando las estructuras de datos que usa para representar estados y decisiones y el formato en el que su algoritmo expresará la respuesta). 
\item Escriba la colección de decisiones óptimas que encontró al ejecutar su algoritmo.
\end{enumerate}


\item ({\it Programación entera})  Resuelva el siguiente problema:
\[
\text{Minimizar } -x_1+x_2
\]
sujeto a las restricciones: 
\[
\begin{aligned}
12x_1+11x_2 &\leq 63, \\
-22x_1+4x_2 &\leq -33, \\
x_1, x_2 &\geq 0, \\
x_1, x_2 &\in \mathbb{Z}.
\end{aligned}
\]
llevando a cabo los siguientes pasos

\begin{enumerate}
\item En el plano $x_1,x_2$ Dibuje la región factible dada por
\[
\begin{aligned}
12x_1+11x_2 &\leq 63, \\
-22x_1+4x_2 &\leq -33, \\
x_1, x_2 &\geq 0, \\
\end{aligned}
\]
y en su interior marque los puntos con coordenadas enteras. Viendo su dibujo, cuál es el valor óptimo del problema?

\item Ahora resuelva el problema mediante Branch-and-Bound. Debe:
\begin{enumerate}
\item Dibujar el árbol de exploración. En cada nodo debe aclarar cuál es el problema del nodo, calcular un upper bound mediante programación lineal y un lower bound (cualquiera de los bounds puede ser $\pm \infty$ reflejando que no se tiene información ó que el problema es infactible). 
\item Para calcular las cotas superiores debe adaptar el código en python que vimos en clase para resolver problemas de optimización lineal.
\item Debe justificar cuándo se detiene el branch-and-bound y cómo el algoritmo garantiza haber alcanzado una solución óptima.

\end{enumerate}

 
\end{enumerate}






\item ({\it Un algoritmo genético innecesario}) Sea $X$ el conjunto de cadenas binarias de longitud $5$. Cada cadena de estas es la representación binaria de un número entero entre $0$ y $32$ inclusive. Sea $f(x)=x^3$. Usaremos algoritmos genéticos para maximizar la función $f$ en el conjunto $X$.
\begin{enumerate}
\item Escriba el código de un algoritmo genético que intente maximizar la función $f(x)$ para $x\in X$. Su código debe incluir funciones de \verb!reproducción!, de \verb!cross-over! y de \verb!mutación! donde la última depende de una probabilidad de mutación $q$ a ser especificada por el usuario.  

\item Construya una población inicial aleatoria $I_0$ de $4$ cadenas. Ejecute su algortimo un total de $10$ pasos y produzca una sola imagen con las gráficas de los valores mínimo, promedio y máximo de la función objetivo en su población (el eje $x$ de su dibujo es el tiempo y el eje $y$ las unidades de la función objetivo y debe contener tres curvas, la del mínimo, la del promedio y la del máximo).

\item Recalcule el dibujo de la parte $(b)$ quitandole a su algoritmo la fase de \verb!cross-over!. Como cambian sus resultados?

\item Recalcule el dibujo de la parte $(b)$ quitandole a su algoritmo la fase de \verb!mutación!. Como cambian sus resultados?

\end{enumerate}
 
\end{enumerate}
\end{document}
