
\documentclass[12pt, a4paper]{article}
\usepackage{hyperref}
\hypersetup{
  colorlinks=true,
  linkcolor=blue,
  urlcolor=cyan,
}
\urlstyle{same}
\usepackage[utf8]{inputenc}
\usepackage{amsmath}
\usepackage{amsfonts}
\usepackage{amssymb}
\usepackage{graphicx}


\newtheorem{theorem}{Teorema.}
\newtheorem{lemma}[theorem]{Lema.}
\newtheorem{corollary}[theorem]{Corolario.}
\newtheorem{definition}[theorem]{Definici\'on:}
\newtheorem{example}[theorem]{Ejemplo:}
\newtheorem{problema}[theorem]{Problema:}
\newtheorem{remark}[theorem]{Observaci\'on:}

\usepackage{graphicx}
\usepackage[spanish]{babel}
%\usetheme{default}

\newcommand{\pp}{\mathbb{P}}
\newcommand{\zz}{\mathbb{Z}}
\newcommand{\rr}{\mathbb{R}}
\newcommand{\qq}{\mathbb{Q}}

\usepackage{tikz, tikz-3dplot}

\definecolor{cof}{RGB}{219,144,71}
\definecolor{pur}{RGB}{186,146,162}
\definecolor{greeo}{RGB}{91,173,69}
\definecolor{greet}{RGB}{52,111,72}

\date{}

\begin{document}
\title{Pr\'actico ALGABO: Más programación dinámica y el algoritmo $A^*$.}
\maketitle{}
\begin{enumerate}
\item ({\it Floyd Warshall}). 
\begin{enumerate}
\item Describa con palabras qué problema resuelve el algoritmo de Floyd Warshall.

\item Encuentre una recursión de Bellman que le permita resolver el problema mediante programación dinámica.

\item Escriba el código de su implementación de la solución del problema.
\item Utilice su implementación para resolver el problema en el grafo con matriz de adyacencia
\[
\begin{array}{c|cccccc}
    & A & B & C & D & E & F \\
\hline
A & 0 & 4 & 1 & 3 & \infty & \infty \\
B & \infty & 0 & 2 & \infty & 5 & \infty \\
C & \infty & \infty & 0 & 8 & 7 & \infty \\
D & \infty & \infty & \infty & 0 & 2 & 4 \\
E & \infty & \infty & \infty & \infty & 0 & 6 \\
F & \infty & \infty & \infty & \infty & \infty & 0 \\
\end{array}
\]
\end{enumerate}


\item ({\it Árboles de búsqueda óptimos}) Tenemos una lista de items $1,2,\dots, n$ con frecuencias respectivas de búsqueda $p_1,\dots, p_n$.
\begin{enumerate}
\item Escriba formalmente el problema de optimización que quiere resolverse para encontrar un \emph{\'arbol de b\'usqueda \'optimo}.
\item Sea $W_{i,j}$ el tiempo esperado (weighted time) de búsqueda de un árbol binario óptimo de los items $\{i,i+1,\dots, j\}$ con frecuencias respectivas de búsqueda $p_i,p_{i+1},\dots, p_j$. Encuentre la recursión de Bellman que satiface $W_{i,j}$ justificando detalladamente su respuesta.
\item Escriba su implementación de un algoritmo que calcule los valores $W_{ij}$ de manera recursiva.
\item Utilice su implementación para calcular el tiempo esperado de búsqueda y para encontrar el árbol óptimo para los siguientes siete items. Escriba los resultados intermedios utilizados en la evaluación de $W_{ij}$.
\begin{center}
\begin{tabular}{c|c}
item & Frecuencia\\
1 & 20\\
2 & 5\\
3 & 17\\
4 & 10\\
5 & 20\\
6 & 3\\
7 & 25\\
\end{tabular}
\end{center}
\item Cuántos árboles binarios (no necesariamente balanceados) posibles hay en $7$ items?

\end{enumerate}

\item ({\it Aplicando $A^*$ a mano}) Considere el grafo con vértices $A,B,C,D,E,G$ con vértice de inicio $s=A$, vértice de salida $t=G$ y aristas dadas por la siguiente tabla:
\begin{center}
\begin{tabular}{c|c}
Arista & Peso\\
(A,B) & 1\\
(A,C) & 3\\
(B,C) & 3\\
(B,D) & 4\\
(C,E) & 1\\
(D,G) & 5\\
(E,G) & 2\\
\end{tabular}
\end{center}
En los siguientes ejercicios deberá ejecutar diferentes versiones del algoritmo $A^*$ en este grafo a mano. Debe escribir los valores de $X,Q$ $\phi$ y ${\rm prev}$ en cada uno de los ciclos de ejecución. No olvide dibujar el árbol de búsqueda tal como hicimos en clase.

\begin{enumerate}
\item Ejecute el algoritmo de Dijkstra (es decir $A^*$ con la heurística $h=0$).
\item Ejecute el algoritmo $A^*$ con la heurística
\[h(v)=\begin{cases}
4\text{, si $v=D$}\\
0\text{, de lo contrario}
\end{cases}
\]
\item Defina {\it heurística admisible} y demuestre que la heurística del numeral anterior es admisible.
\item Ejecute el algoritmo $A^*$ a  con la heurística
\[h(v)=\begin{cases}
100\text{, si $v=C$}\\
0\text{, de lo contrario}
\end{cases}
\]
\item Es la heurística del numeral anterior admisible? Justifique su respuesta.
\end{enumerate}

\item Suponga que alguien nos dá la heurística perfecta, es decir $\overline{h}(v)$ es exáctamente igual al mínimo costo de un camino que sale de $v$ y llega a algún vértice de salida. Cómo se comporta el algoritmo de $A^*$ con la heurística $\overline{h}(v)$? Justifique rigurosamente su respuesta. 


\end{enumerate}
\end{document}


